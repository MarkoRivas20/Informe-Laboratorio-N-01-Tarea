\section{Introducción} 
\begin{flushright}


\begin{itemize}
ORM es el mapeo objeto-relacional (más conocido por su nombre en inglés, Object- Relational mapping), consiste en una técnica de programación para convertir datos entre el lenguaje de programación orientado a objetos utilizado y el sistema de base de datos relacional utilizado en el desarrollo de aplicaciones. Esto posibilita el uso de las características propias de la orientación a objetos (básicamente herencia y polimorfismo). Hay paquetes comerciales y de uso libre disponibles que desarrollan el mapeo relacional de objetos, aunque algunos programadores prefieren crear sus propias herramientas ORM. \textbf{}\\
\textbf{}\\
Entre estos paquetes comerciales tenemos una lista alfabética de los principales motores de mapeo objeto relacional, tales como: ColdFusion, Common Lisp, Java, JavaScript, .NET, Perl, PHP, Python, Ruby, Smalltalk, C++.\textbf{}\\
\textbf{}\\
El problema que surge, porque hoy en día prácticamente todas las aplicaciones están diseñadas para usar la Programación Orientación a Objetos (POO), mientras que las bases de datos más extendidas son del tipo relacional y estas solo permiten guardar tipos de datos primitivos (enteros, cadenas de texto) por lo que no se puede guardar de forma directa los objetos de la aplicación en las tablas, sino que estos se deben de convertir antes en registros, que por lo general afectan a varias tablas. En el momento de volver a recuperar los datos, hay que hacer el proceso contrario, se deben convertir los registros en objetos. \textbf{}\\
\textbf{}\\
En este punto es que se muestra la importancia del ORM, ya que este se encarga de forma automática, de convertir los objetos en registros y viceversa, simulando así tener una base de datos orientada a objetos.




	


\end{itemize} 


\end{flushright}